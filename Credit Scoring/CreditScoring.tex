\documentclass[10pt, oneside]{article} 
\usepackage{amsmath, amsthm, amssymb, calrsfs, wasysym, verbatim, bbm, color, graphics, geometry}

\geometry{tmargin=.75in, bmargin=.75in, lmargin=.75in, rmargin = .75in}  

\newcommand{\R}{\mathbb{R}}
\newcommand{\C}{\mathbb{C}}
\newcommand{\Z}{\mathbb{Z}}
\newcommand{\N}{\mathbb{N}}
\newcommand{\Q}{\mathbb{Q}}
\newcommand{\Cdot}{\boldsymbol{\cdot}}

\newtheorem{thm}{Theorem}
\newtheorem{defn}{Definition}
\newtheorem{conv}{Convention}
\newtheorem{rem}{Remark}
\newtheorem{lem}{Lemma}
\newtheorem{cor}{Corollary}


\title{Running Lecture Outline: [Course Code]}
\author{[Sitraka FORLER]}
\date{Academic Year 2022-2023}

\begin{document}

\maketitle
\tableofcontents

\vspace{.25in}

\section{Fall 2019}

\subsection{Bassel 1 to 4 }

\begin{itemize}

\item Newtonian mechanics (i.e., ${\bf F} = m{\bf a}$) is an excellent theory; it applies to the vast majority of human-scale (and even interplanetary-scale) physics. 

\item Apart from relativistic effects at very high velocities (special relativity) or in very strong gravitational fields (general relativity), Newtonian mechanics accurately describes a huge range of phenomena, but around the end of the Nineteenth Century people became aware of some physical effects for which there is no sensible Newtonian explanation.

\item Examples include:
\begin{itemize}
\item the {\bf double slit experiment} (done with light by Thomas Young in 1801, and with electrons by Tonomura in 1986)
\item the photoelectric effect (analyzed by Einstein in 1905 --- in fact his Nobel-winning work)
\item the ``quantum Venn diagram'' puzzle, involving the overlaps of three polarizing filters
\item the stability of the hydrogen atom (i.e., the fact that the electron doesn't lose energy and spiral inward toward the proton).
\end{itemize}

\begin{rem}
How now, brown cow?
\end{rem}

\begin{defn}
The {\em Feynman kernel} is given by
\[ K(x_b, t_b; x_a, t_a) = \int_{x(t_a) = x_a}^{x(t_b) = x_b} e^{(i/\hbar) S[x(t)]} \; \mathcal{D}x(t). \]
\end{defn}

\end{itemize}

\subsection{Mon, Sept 9: Review of Newtonian Mechanics}

\begin{itemize}

\item  A {\em Newtonian trajectory} ${\bf x}(t)$ ($t \in \R$) is given by solutions of the second order ODE
\[ m\, \ddot{{\bf x}}(t) = {\bf F}({\bf x}(t)), \]
where $m > 0$ is a basic parameter associated with a given Newtonian particle, called its {\em mass}.

\item The force field ${\bf F}({\bf x})$ --- which we take to be static (i.e., not intrinsically dependent on time) for simplicity --- is said to be {\em conservative} if there is a {\em potential function} $V({\bf x})$ such that
\[ {\bf F}({\bf x}) = -\nabla V({\bf x}). \]
Here, `$\nabla$' denotes the {\em gradient operator}, 
\[ \nabla V = \bigg( \frac{\partial V}{\partial x}, \, \frac{\partial V}{\partial y}, \, \frac{\partial V}{\partial z} \bigg). \]

\item For a conservative force field, we can find a {\em conserved quantity} along the Newtonian trajectories, namely the {\em total mechanical energy}.
\[ E = H({\bf x}, {\bf p}) := \frac{1}{2m}\, {\bf p}^2 + V({\bf x}). \]
Here, ${\bf p}^2 := {\bf p} \Cdot {\bf p} = \| {\bf p} \|^2$, and ${\bf p} := m {\bf v} := m \dot{{\bf x}}$ is the {\em momentum}.

\end{itemize}

\subsection{Tue, Sept 10: Alternative Formulations of Newtonian Mechanics}

\begin{itemize}

\item The {\bf Hamiltonian formulation}:
\[ \dot{{\bf x}} = \frac{\partial H}{\partial {\bf p}}, \;\;\;\;\; \dot{{\bf p}} = - \frac{\partial H}{\partial {\bf x}}. \]

\item The {\bf Lagrangian formulation}:
\[ \delta S[{\bf x}(t)] = 0, \]
where the {\em action} on the time interval $[t_a, t_b]$ is given by
\[ S[{\bf x}(t)] := \int_{t_a}^{t_b} \bigg[ \frac{m}{2} \, \dot{{\bf x}}(t)^2 - V({\bf x}(t)) \bigg]  \; dt. \]

\item Etc.

\end{itemize}


\section{Spring 2020}



\end{document}
